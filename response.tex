\documentclass{response}
\usepackage{lipsum}

\def\author{Kakei \textit{et al.}}
\def\title{\lipsum[1][1]}
\begin{document}

% Let's start point-by-point with Reviewer 1
\reviewersection{Reviewer}{1}

% Point one description 
\begin{point}
	\lipsum[1]
\end{point}

% Our reply
\begin{response}
	We agree with the reviewer on this important point. This is what we did to
	fix it. 
	\lipsum[2]
\end{response}

\begin{action}
    \lipsum[1]
    
    \lipsum[1]
    
    \begin{change}[Section 1]
    	\lipsum[2]
     
    	\lipsum[3]
    \end{change}
    
    \lipsum[4]
    
    \begin{change}[Section 2]
    	\lipsum[5]
     
    	\lipsum[6]
    \end{change}
\end{action}

\begin{point}
	\Reviewer's second point.
	\label{pt:bar}
\end{point}

\begin{response}
	And our reply to it.
\end{response}

% Begin a new reviewer section
\reviewersection{Reviewer}{2}

\begin{point}
	This is the first point of Reviewer \Reviewer. With some more words foo
	bar foo bar ...
\end{point}

\begin{response}
	Our reply to it with reference to one of our points above using the \LaTeX's 
	label/ref system (see also \ref{pt:foo}).
\end{response}

\end{document}