\documentclass{cover-letter}
%\usepackage[%
%  margin=0.5in,
%  headsep=3mm, % <======================================================
%  marginparsep=3mm, % <=================================================
%  showframe
%]{geometry}

%\topmargin=-1in    % Make letterhead start about 1 inch from top of page 
%\textheight=8in  % text height can be bigger for a longer letter
%\oddsidemargin=0pt % leftmargin is 1 inch
%\textwidth=6.5in   % textwidth of 6.5in leaves 1 inch for right margin

\def\name{First Last}
\signature{\name}

%\def\phone{(xxx) xxx-xxxx}
\def\email{email@email.com}
\def\LinkedIn{timmy-l-chan} % linkedin.com/in/______
\def\github{TimmyChan} % github username
\def\role{Data Scientist} % JOB TITLE

% ============================
% Editor & Journal Info.
% ============================
\def\EditorFirstName{John}
\def\EditorLastName{Smith}
\def\EditorName{\EditorFirstName~\EditorLastName}
\def\EditorTitle{Editor-in-Chief, Managing Editor, Co-Editors-in-Chief}
\def\JournalName{Journal of XXX YYY ZZZ} % position
\def\JournalAddress{Address 1} % company

%[Journal Editor’s First and Last Name][, Graduate Degree (if any)]
%TIP: It’s customary to include any graduate degrees in the addressee’s name.
%e.g., John Smith, MD or Carolyn Daniels, MPH
%[Title]
%e.g., Editor-in-Chief, Managing Editor, Co-Editors-in-Chief
%[Journal Name]
%[Journal Address]
%[Submission Date: Month Day, Year]

% =========================
% Paper Info.
% =========================
\def\PaperTitle{PAPER TITLE}
\def\YourName{YOUR NAME}
\def\YourEmail{YOUR Email}
\def\YourTitle{YOUR TITLE}
\def\InstitutionName{INSTITUTION NAME}
\def\InstitutionAddress{INSTITUTION ADDRESS}

\begin{document}

%\begin{letter}{\jname \\ \jtitle \\ \jname \\ \jaddress  }
\begin{letter}{\EditorName \\ \EditorTitle \\ \JournalName \\ \JournalAddress}

\opening{Dear Dr./Mr./Ms. \EditorLastName:}

% TIP
% - Where the editor’s name is not known,
%   use the relevant title employed by the journal,
%   such as “Dear Managing Editor:” or “Dear Editor-in-Chief:”.
%   Using a person’s name is best, however.
% - Use “Ms.” and never “Mrs.” or “Miss” in formal business letters.
% - Never use “Dear Sirs:” or any similar expression.
%   Many editors will find this insulting,
%   especially given that many of them are female!

% ====================================================
% [Para.1: 2–3 sentences]
% ====================================================

% [TIP]
% 発見と結論について議論するのに便利なフレーズには、以下のようなものがあります。
%   - Our findings confirm that…
%   - We have determined that…
%   - Our results suggest…
%   - We found that…
%   - We illustrate…
%   - Our findings reveal…
%   - Our study clarifies…
%   - Our research corroborates…
%   - Our results establish…
%   - Our work substantiates…

% [Example]
% I am writing to submit our manuscript entitled, “X Marks the Spot” for consideration as an Awesome Science Journal research article. We examined the efficacy of using X factors as indicators for depression in Y subjects in Z regions through a 12-month prospective cohort study and can confirm that monitoring the levels of X is critical to identifying the onset of depression, regardless of geographical influences.

I am writing to submit our manuscript entitled, ``\PaperTitle'' for consideration as an \JournalName.

% ====================================================
% [Para. 2: 2–5 sentences]
% ====================================================

% [TIP]
% ジャーナルの典型的な読者を特定し、その人々がトピックに対する理解を深めるために、あなたの研究をどのように活用できるかを説明する。例えば、対象となる雑誌の読者の多くが、さまざまな調査研究の公共政策への影響に関心を持っている場合、あなたの結論が、公共の懸念により効果的に対処する、より強力な政策を開発するために、同業者にどのように役立つかを議論することができるでしょう。

% [Example]
% Given that [context that prompted your research], we believe that the findings presented in our paper will appeal to the [Reader Profile] who subscribe to [Journal Name]. Our findings will allow your readers to [identify the aspects of the journal’s Aim and Scope that align with your paper].

% [TIP]
% なぜこの研究課題に取り組まなければならないのか、その背景を含めること。

% [Example]
% “Given the struggle policymakers have had to define proper criteria to diagnose the onset of depression in teenagers, we felt compelled to identify a cost-effective and universal methodology that local school administrators can use to screen students.”

% [TIP]
% 論文が先行研究によって促されたものである場合、その旨を明記する。例えば、「最初にXを研究した後、YからZを調査するフォローアップ研究を行うよう打診されました。このプロジェクトを進めている間に、私たちは(出版を通じて仲間と情報を共有する必要があると判断した何らかの新しい理解)を発見しました」。

% [Example]
% Given the alarming increase in depression rates among teenagers and the lack of any uniform practical tests for screening students, we believe that the findings presented in our paper will appeal to education policymakers who subscribe to The Journal of Education. Although prior research has identified a few methods that could be used in depression screening, such as X and Y, the applications developed from those findings have been cost-prohibitive and difficult to administer on a national level. Thus, our findings will allow your readers to understand the factors involved in identifying the onset of depression in teenagers better and develop more cost-effective screening procedures that can be employed nationally. In so doing, we hope that our research advances the toolset needed to combat the concerns preoccupying the minds of many school administrators.

% ====================================================
% [Para 3: Similar works]
% ====================================================

% [TIP]
% 対象ジャーナルが最近発表した類似の研究があれば言及する必要がありますが、その数は5つまでとします。1つの論文にしか言及しない場合は、前文を "This paper [examines a different aspect of] / [takes a different approach to] the issues explored by [Authors] in [Article Title], also published by [Journal Name] on [DATE]." に置き換える。

% [Example]
% “This manuscript expands on the prior research conducted and published by [Authors] in [Journal Name]” or “This paper [examines a different aspect of]/ [takes a different approach to] the issues explored in the following papers also published by [Journal Name].”
%
%    Article 1
%    Article 2
%    Article 3


% ====================================================
% [Para. 4: Additional statements often required]
% ====================================================

% [Example]
% Each of the authors confirms that this manuscript has not been previously published and is not currently under consideration by any other journal. Additionally, all of the authors have approved the contents of this paper and have agreed to the [Journal Name]’s submission policies.

% [TIP]
% 過去に何らかの形で、あるいは研究の一部を他の場所で公開したことがある場合は、その旨を明記します。例えば、"We have presented a subset of our findings [at Event] / [as a Type of Publication Medium] in [Location] in [Year]" と記載することができます。

% [Example]
% We have since expanded the scope of our research to contemplate international feasibility and acquired additional data that has helped us to develop a new understanding of geographical influences.

% ====================================================
% [Para. 5: Potential Reviewers]
% ====================================================

% [Example]
% Should you select our manuscript for peer review, we would like to suggest the following potential reviewers/referees because they would have the requisite background to evaluate our findings and interpretation objectively.
% 
%     [Name, institution, email, expertise]
%     [Name, institution, email, expertise]
%     [Name, institution, email, expertise]
% 
% To the best of our knowledge, none of the above-suggested persons have any conflict of interest, financial or otherwise.

% [TIP]
% ジャーナルがあなたの提案のうち少なくとも1人を使用する可能性があるため、3~5人の査読者を含めること。

% [TIP]
% 投稿先のジャーナルが使用する用語(「reviewer」または「referee」)のいずれかを使用します。ジャーナルの用語に注意を払うことは、そのジャーナルをきちんとリサーチし、準備した証です

% ====================================================
% [Para. 6: Frequently requested additional information]
% ====================================================
Each named author has substantially contributed to conducting the underlying research and drafting this manuscript. Additionally, to the best of our knowledge, the named authors have no conflict of interest, financial or otherwise.

Sincerely,

\YourName

Corresponding Author\\
\YourTitle\\
\InstitutionName\\
\InstitutionAddress\\
\YourEmail

\end{letter}
\end{document}